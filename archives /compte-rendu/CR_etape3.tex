\documentclass[a4paper,10pt]{article}
\usepackage[utf8]{inputenc}
\usepackage{amssymb}
\usepackage{amsmath}
\usepackage{graphicx}
\newcommand{\HRule}{\rule{\linewidth}{0.5mm}}
\usepackage{subfigure}
\usepackage{multicol}
\usepackage[usenames,dvipsnames]{color}
\definecolor{darkgray}{rgb}{0.95,0.95,0.95}
\usepackage{listings}
\usepackage{color}
\usepackage{textcomp}
\definecolor{listinggray}{gray}{0.9}
\definecolor{lbcolor}{rgb}{0.9,0.9,0.9}
\lstset{
    backgroundcolor=\color{lbcolor},
    tabsize=4,
    rulecolor=,
    language=C++,
    basicstyle=\scriptsize,
    upquote=true,
    aboveskip={1.5\baselineskip},
    columns=fixed,
    showstringspaces=false,
    extendedchars=true,
    breaklines=true,
    prebreak = \raisebox{0ex}[0ex][0ex]{\ensuremath{\hookleftarrow}},
    frame=single,
    showtabs=false,
    showspaces=false,
    showstringspaces=false,
    identifierstyle=\ttfamily,
    keywordstyle=\color[rgb]{0,0,1},
    commentstyle=\color[rgb]{0.133,0.545,0.133},
    stringstyle=\color[rgb]{0.627,0.126,0.941},
}
\lstset{language=C++}
\lstset{backgroundcolor=\color{darkgray}}

\usepackage[left=1.0cm, right=1.0cm, top=2cm, bottom=4cm]{geometry}
\usepackage{fancyhdr}
\pagestyle{fancy}
\usepackage{lastpage}
\renewcommand\headrulewidth{1pt}
\fancyhead[L]{\textsc{Nachos Étape 3 : Multithreading}}
\fancyhead[R]{\textsc{Polytech' Grenoble}}	
\renewcommand\footrulewidth{1pt}
\fancyfoot[R]{ \textsc{RICM 4}}

\begin{document}
\begin{titlepage}

\begin{center}


% Upper part of the page
\includegraphics[width=0.25\textwidth]{../Desktop/POLYTECH_RVB.jpg}\\[1cm]

\textsc{\LARGE Polytech' Grenoble}\\[1.5cm]

\textsc{\Large RICM 4ème année}\\[1.2cm]


% Title
\HRule \\[0.4cm]
{ \huge \bfseries NachOS\\[0.6cm]
Etape 3: Multithreading}
\\[0.4cm]

\HRule \\[2cm]

% Author and supervisor
\begin{minipage}{0.4\textwidth}
\begin{flushleft} \large
\emph{\'Etudiants:}\\
Paul \textsc{LABAT} \\
Rodolphe \textsc{FREBY} \\
Kai \textsc{GUO} \\
Zhengmeng \textsc{ZHANG}
\end{flushleft}
\end{minipage}
\begin{minipage}{0.4\textwidth}
\begin{flushright} \large
\emph{Professeur:} \\
Jean-françois \textsc{MEHAUT}
\end{flushright}
\end{minipage}

\vfill

% Bottom of the page
{\large  Mars 2014}

\end{center}

\end{titlepage}

\section{Applications multi-thread}

\subsection{La classe UserThread}
Pour implémenter les threads utilisateurs nous avons créé une nouvelle classe
UserThread. Cet objet va servir également à stocker les arguments qu'on passe à la fonction UserThreadCreate().

\textit{code/userprog/userthread.h}
\begin{lstlisting}
typedef struct{
	int f;
	int arg;
}argThreadUser;

extern int do_UserThreadCreate(int f, int arg);
extern void do_UserThreadExit();
extern void do_UserThreadJoin(int i);
\end{lstlisting}

\textit{code/userprog/userthread.cc}
\begin{lstlisting}
static void StartUserThread(int f){

  	//Recuperation des parametres passes precedements
  	argThreadUser *argRecup = (argThreadUser *) f;


    //Initialisation des registres
    // Mise en place le PC correctement, en debut de fonction
    machine->WriteRegister (PCReg, argRecup->f);
    // Mise place ensuite l'argument dans le registre 4
    machine->WriteRegister(4, argRecup->arg);
    
    // Mise en place du PC Next pour l'instruction suivante
    machine->WriteRegister (NextPCReg, argRecup->f+4);

    // Initialisation de la bitmap 
	machine->WriteRegister(StackReg, currentThread->space->InitRegistersU(&(currentThread->id)));
  	

    // Lancement 
    currentThread->space->addThread();
    machine->Run();
}
\end{lstlisting}

\subsection{Modifications de l'espace d'adressage du processus}

Pour gérer plusieurs threads, nous sommes obligés de modifier l'espace
d'adressage (AddrSpace) du processus. Nous avons entre autres ajouté le nombre
de threads en exécution, un objet bitmap pour gérer l'allocation des
zones des threads et des sémaphores pour utiliser ces attributs en section
critique. Chaque thread a sa propre pile qui est une partie de la pile du
processus (sa taille est de 3 pages). Nous sommes donc naturellement limités en
nombre de threads simultanés par processus. Lorsque cette limitation
se manifeste, on renvoie simplement un code d'erreur (-1).

\textit{code/userprog/addrspace.cc}
\begin{lstlisting}
int AddrSpace::InitRegistersU(int *threadId) {
    
    int startStack = numPages*PageSize;

    int renvoyer = -1;
    semBM->P();   
    //*threadId = bitmap->Find();
    if(*threadId != -1) {
        renvoyer = startStack - (PageSize*MAX_PAGE_THREADS*(*threadId));
    }
    semBM->V();
    return renvoyer;

}

// lors de la fin d'un thread

void AddrSpace::deleteThread(){

    semT->P();
    semBM->P();

    nbT--;
    //Suppression du thread de la bitmap
    bitmap->Clear(currentThread->id);
    if(nbT == 0){
        //nous n'avons plus aucun autre thread que le thread principal, on peut donc l'arreter
        semA->V();
    }
    semT->V();
    semBM->V();
}

// lors de l'ajout d'un thread

void AddrSpace::addThread(){

    semT->P();
    nbT++;
    semT->V();
  
}
\end{lstlisting}

\subsection{Mécanisme d'attente des threads (UserThreadJoin)}

Nous utilisons un tableau de sémaphore de la taille du nombre maximal de threads.
Chaque thread (A) qui veut se bloquer sur un autre (B) va devoir prendre un jeton
du sémaphore lié au thread B sur lequel il veut se bloquer. Ces sémaphores sont
donc bloquants lorsque le thread B est en cours d'exécution.

Une fois terminé, le thread B va réveiller les threads qui se sont bloqués que
lui en déposant un jeton dans le sémaphore concerné.
Pour permettre à plusieurs threads de se bloquer sur le même thread, la phase
de réveil doit se faire en chaîne, c'est-à-dire chaque thread réveillé doit
réveiller le suivant etc..

Ce mécanisme doit également bloquer le prochain thread qui va s'exécuter dans
la nouvelle zone liberée par le thread B.

Pour arriver à ce résultat, nous initialisons le tableau de sémaphore à un jeton
chacun. A chaque création, il faut prendre un jeton (pour que le sémaphore
devienne bloquant pour tous les threads qui appellent \textbf{UserThreadJoin()}).
S'il y a une phase de réveil qui est en cours, ce nouveau thread sera bloqué,
et c'est ce qu'on cherche pour pas qu'un thread bloqué sur un thread spécifique
se retrouve bloqué sur d'autres (avec le risque qu'il ne se réveille jamais)

\textit{code/test/userthread.cc}
\begin{lstlisting}
int do_UserThreadCreate(int f, int arg){
	Thread *newThread = new Thread("new thread");
	// Si le thread cree est nul, on renvoie -1
	if(newThread == NULL){
		return -1;
	}

	// Arguments concatenes
	argThreadUser * argv = new argThreadUser;
	argv->f = f;
	argv->arg = arg;

    newThread->id = currentThread->space->getID();
    if(newThread->id == -1){
        // Une erreur
        printf("\nError during the creation of a new thread\n");
        delete newThread;
    }else{
    	// Placage en liste d'attente
    	newThread->Fork(StartUserThread, (int)argv);
    }
    currentThread->space->semJoin[newThread->id]->P();

    //recuperation de l'id du thread !
    machine->WriteRegister(2, newThread->id);
    // Sans cela, obligation d'effectuer une action  dans le main, car les threads sont justes crees mais
    // pas encore demarres
    currentThread->Yield();
	return 0;
}
\end{lstlisting}

\begin{lstlisting}
void do_UserThreadExit() {
		currentThread->space->semJoin[currentThread->id]->V();
        currentThread->space->deleteThread();
        currentThread->Finish();
}
\end{lstlisting}

\begin{lstlisting}
void do_UserThreadJoin(int i){

    currentThread->space->semJoin[i]->P();
}
\end{lstlisting}

\section{Test de l'étape 3}

Vous pouvez soit lancer le script de test automatique \textit{runtest.sh} soit
lancer manuellement les deux programmes \textit{./build-origin/nachos-userprog
-rs 1 -x ./build/jointhreads} et \textit{./build-origin/nachos-userprog -rs 1 -x
./build/makethreads}

\textit{code/test/userthread.cc}
\begin{lstlisting}
#include "syscall.h"

void print(int i){
	int j = 0;
	for(j = 0; j < 3; j++){
		if(i%2 == 0){
			SynchPutString("Coucou, je m'execute dans un thread avec un param pair !");
		}else{
			SynchPutString("Coucou, je m'execute dans un thread avec un param impaire !");
		}
	}
	UserThreadExit();
}

int main(){
	int t1, t2;
	t1 = UserThreadCreate(print,(void *) 7);
	t2 = UserThreadCreate(print,(void *) 8);
	return 0;
}
\end{lstlisting}

\begin{lstlisting}
$ ./nachos-userprog -rs 1 -x makethreads
Coucou, je m'execute dans un thread avec un param impaire !
Coucou, je m'execute dans un thread avec un param impaire !
Coucou, je m'execute dans un thread avec un param impaire !
Coucou, je m'execute dans un thread avec un param pair !
Coucou, je m'execute dans un thread avec un param pair !
Coucou, je m'execute dans un thread avec un param pair !
Machine halting!

Ticks: total 47138, idle 34966, system 11960, user 212
Disk I/O: reads 0, writes 0
Console I/O: reads 0, writes 351
Paging: faults 0
Network I/O: packets received 0, sent 0

Cleaning up...
\end{lstlisting}

Sans l'option -rs le thread main ne donne jamais la main aux autres
threads. Comme on peut le voir sur le test, le nombre de threads simultanés
est limité (par la taille de l'espace d'adressage et la taille de la
pile d'un thread). Dans ce cas un code d'erreur (-1) est renvoyé lors de
l'appel à \textit{UserThreadCreate}.


\end{document}